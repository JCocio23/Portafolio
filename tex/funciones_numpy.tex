\documentclass[../portafolio.tex]{subfiles}
\begin{document}


\chapter{Análisis de Funciones en Numpy y Resolución de problemas}
\label{ch:analisis_funciones_numpy}

\chapterauthor{Ignacio Falcón, Joaquín Parra, Alvaro Osses}

\hfill \textbf{Fecha de la actividad:} 8 de Octubre del 2025

Para este capítulo, se investigó y trabajó con una variedad de funciones pertenecientes a la libreria \textit{Numpy} del lenguaje de Programación Python con el objetivo de evaluar y comprobar las capacidades que pueden brindar a la hora de trabajar en matemáticas. Luego, se analizaron las funciones estudiadas con el objetivo de demostrar su utilidad frente a objetivos concretos.

\section {Funciones de Numpy}

Para esta sección, se tenía como problema:

\textit{Consultar y buscar información acerca de las siguientes funciones de }\lstinline|numpy|, \lstinline|arange, array, linspace,|
\lstinline |geomspace, ones, zeros, random.random, random.normal, random.randint|, \textit{además, se debe listar cada función según su nombre, definición y ejemplo. }

Para esto, se confeccionó una tabla compuesta por un conjunto de múltiples funciones pertenecientes a la librería \textit{Numpy} junto a su debida explicación, a su vez, se listó ejemplos del uso de cada una de las funciones mencionadas.

\begin{table} 
    \begin{tabular}{l c c }
    Función & Explicación & Ejemplos \\ 
    \hline
    
    \parbox[t]{2.5cm}{arange} & \parbox[t]{8cm}{Función que genera un arreglo númerico en un intervalo a saltos, debe indicarse en orden, el inicio y término del intervalo (no lo toma) y el salto entre valores.} & \parbox[t]{4cm}{\lstinline|numpy.arange(0,10,2)|    \lstinline|# array([0. 2. 4. 6. 8.])|} \\ 
    && \\
    \parbox[t]{2.5cm}{array} & \parbox[t]{8cm}{Función que convierte una lista específica de datos en un arreglo numérico de Numpy, se le debe indicar dentro del paréntesis la lista a trabajar.} & \parbox[t]{4cm}{\lstinline|numpy.array([1,2,3])|   \lstinline|#array([1. 2. 3.])|} \\ 
    && \\
    \parbox[t]{2.5cm}{linspace} & \parbox[t]{8cm}{Función que genera un arreglo en un intervalo para una cantidad de puntos a una distancia igual, se debe indicar en orden donde comienza el intervalo, donde termina el intervalo y la dimensión de datos del arreglo.} & \parbox[t]{4cm}{\lstinline|numpy.linspace(0,10,6)|    \lstinline|#array([0. 2. 4. 6. 8. 10.])|} \\ 
    && \\
    \parbox[t]{2.5cm}{geomspace} & \parbox[t]{8cm}{Función que genera un arreglo en un intervalo a una proporción geométrica entre valores, se le debe indicar en orden donde comienza el intervalo, donde termina y la dimensión de datos que tendrá el arreglo. } &\parbox[t]{4cm}{ \lstinline|numpy.geomspace(1,1000,4)|     \lstinline|# array([1. 10. 100. 1000.])|}\\ 
    && \\
    \parbox[t]{2.5cm}{ones} & \parbox[t]{8cm}{Función que genera un arreglo de una cantidad específica de coeficientes, los cuales todos son de valor 1, esta cantidad debe ser agregada dentro del paréntesis de la función.} &\parbox[t]{4cm}{\lstinline|numpy.ones(4)|      \lstinline|#array([1. 1. 1. 1.])|} \\ 
    &&\\
    \parbox[t]{2.5cm}{zeros} & \parbox[t]{8cm}{Función que genera un arreglo de una cantidad de específica de coeficientes, los cuales son todos de valor 0, esta cantidad debe ser agregada dentro del paréntesis de la función.} & \parbox[t]{4cm}{\lstinline|numpy.zeros(4)|      \lstinline|# array([0. 0. 0. 0.])|}\\ 
    &&\\ 
    \parbox[t]{2.5cm}{random.random} & \parbox[t]{8cm}{Función que genera un arreglo de una cantidad específica de valores aleatorios que se encuentran entre 0 y 1, este arreglo puede ser de la dimensión que se escoja.} & \parbox[t]{4cm}{\lstinline|numpy.random.random(4)|     \lstinline|#array([0.12771608, 0.65220371|   \lstinline|# , 0.99625441, 0.90112646])|}\\ 
    &&\\
    \parbox[t]{2.5cm}{random.normal} & \parbox[t]{8cm}{Función que genera un arreglo de coeficientes que cumplen una distribución normal en torno a un valor promedio (primer valor), una desviación estándar concreta(Segundo valor) y la dimensión de elementos}& \parbox[t]{4cm}{\lstinline|numpy.random.normal(10,2,5)|   \lstinline|# array([ 8.879 12.321|   \lstinline|# 9.768 11.045  7.533])|}\\ 
    && \\
    \parbox[t]{2.5cm}{random.randint} & \parbox[t]{8cm}{Función que genera un arreglo de valores enteros, en un intervalo definido, debe indicarse en orden el inicio del intervalo, el término el intervalo (valor que no toma) y la dimensión (cantidad) de elementos.} & \parbox[t]{4cm}{\lstinline|numpy.random.randint(1,10, 3) |     \lstinline|# array([3. 7. 9.])|} \\ 
    && \\ \hline \\
    \end{tabular} 
    \caption{Tabla de funciones de Numpy como: Nombre - Descripción - Ejemplo.}
    \label{fig:tabla_funciones_numpy}
\end{table}

Esta tabla se puede encontrar en el Cuadro (\ref{fig:tabla_funciones_numpy})

\section{Visualización de una distribución normal en un conjunto de 200 datos aleatorios} 

Este problema, consistía en generar una cantidad concreta (mayor a 200) de números aleatorios que cumplan con una distribución normal. Luego, en una misma figura, generar un gráfico de $a_n$ vs $n$ y un histograma de $a_n$. Para esto, se abordó el uso de la función \lstinline!numpy.random.normal! para generar los números en cuestión. Luego, haciendo uso de la libreria \textit{MatplotLib.pyplot}, se generaron dos gráficos, cada uno correspondiente a los valores $(n,a_n)$ y al histograma de ${a_n}$.

\begin{lstlisting}
dist_normal = np.random.normal(0,0.5,N) # Se generan N cantidad de números aleatorios 
                                        # según una distribución normal centrada en 0 
                                        # y con desviación estándar de 0.5

plt.scatter(np.arange(0,201), dist_normal) #Crea el gráfico de puntos en el espacio

plt.hist(dist_normal) # Se genera un histograma

\end{lstlisting}

Esto, al ejecutarse, genera la \autoref{fig:histograma_tarea1}, la que compone al gráfico $a_n$ y al histograma de datos ${a_n}$.
\begin{figure}
    \centering
    \includegraphics[width=0.7\textwidth]{../img/analisis_numpy/histograma_tarea1_an.png}
    \caption{Gráfico e Histograma de datos}
    \label{fig:histograma_tarea1}
\end{figure}

\section{Generación de matriz de números complejos e impresión de valores según su posición} 

El problema consistía en definir inicialmente una matriz 3x5 de números complejos, para luego imprimir la primera fila, la última columna y el elemento de la segunda fila y primera columna. 

Para este problema, se generó una matriz concreta haciendo uso de la función array, a la cual se le introdujo 3 listas de 5 valores cada una, todos con números complejos, luego se procedió a imprimir lo requerido. A continuación, el código utilizado.


\begin{lstlisting}
matriz = np.random.random((3,5)) + 1j*np.random.random((3,5)) # Crea una matriz de 3x5 
                        # con números complejos


print(matriz[0]) # Imprime la Primera fila de la matriz
print(matriz[:,-1]) # Imprime la última columna de la matriz
print(matriz[1,0]) # Imprime el coeficiente 2x1 de la matriz

\end{lstlisting}

\section{Diferencia entre elementos adyacentes de dos series de datos numéricos} 

Para el tercer problema, se requería generar una serie enésima de números enteros aleatorios, en este caso, considerando hasta $N=10$, luego, se pedía evaluar una nueva serie de diferencias entre elementos adyacentes de la serie generada. Esto debía trabajarse tanto con ciclos \textit{for}, como con operaciones vectoriales de \textit{Numpy}. 

Para esto, se abordó inicialmente haciendo uso de un ciclo for para generar la diferencia entre valores dentro de arreglos generados por una lista $x_i$, luego, se hizo uso de la función \textit{diff} de Numpy, con la cuál se puede encontrar la diferencia de valor entre posiciones seguidas de un arreglo. A continuación se presenta el código usado.

\begin{lstlisting}
serie = np.random.randint(1,100,n+1) # Crea un array de N+1 números aleatorios enteros 
                                   # del 1 al 100

diferencia = np.zeros(n) # Generamos un arreglo de N espacios con ceros
for i in range(0,len(serie) - 1):
    diferencia[i] = serie[i+1] - serie[i] # Para cada posición del arreglo, evalua 
    # la diferencia entre el espacio siguiente y el actual. 
    
print(serie[1:] - serie[:-1],"Es la serie hecha con la propiedad vectorizada de arrays.") 
# Imprime la nueva serie usando propiedades vectorizadas


\end{lstlisting}

\section{Representación de la función coseno en un gráfico a puntos equidistantes}

Para el último problema, se pedía definir una variable $x$ como un arreglo de 20 elementos en orden creciente en el intervalo $-2\pi \leq x \leq 2\pi $, luego, se requería graficar $\cos x$ en función del arreglo e interpretar su resultado. 

Para esto, se generó una variable x con 20 elementos aleatorios de misma distancia usando la función \textit{linspace}, de Numpy, en un rango entre $[-2\pi,2\pi]$. Luego, haciendo uso de la librería \textit{MatplotLib.pyplot}, se graficó $\cos x$ en función del arreglo x generado anteriormente, como se puede notar en la \autoref{fig:grafico_tarea1_n_vs_an}.

\begin{figure}
    \centering
    \includegraphics[width=0.7\textwidth]{../img/analisis_numpy/grafico_tarea1_n_vs_an.png}
    \caption{Gráfico de cos(x) en función del Arreglo X}
    \label{fig:grafico_tarea1_n_vs_an}
\end{figure}



    Es importante señalar que el gráfico no es suave, en concreto, luce del estilo poligonal con una cantidad N de aristas,  esto es debido a que este está generado en base a una cantidad discreta de valores, en donde \lstinline|plot| genera rectas para cada intervalo entre puntos, esto, a diferencia de la función $\cos x$ para un intervalo, como lo sería para $x = [-2\pi,2\pi]$ donde esta, al poseer una cantidad infinita de puntos, es de gráfica suave. 


    A continuación, se presenta el código usado.
\begin{lstlisting}
plt.figure() # Genera una nueva figura vacía

x = np.linspace(-2*np.pi, 2*np.pi, 20) # Genera un arreglo de 20 valores entre -2pi 
                                       # y 2pi de forma creciente

plt.plot(x, np.cos(x)) # Grafica la función cos(x) con x el arreglo 
                          # anteriormente trabajado

plt.ylabel(r'Valor de $\cos(x)$')
plt.xlabel(r'Valor de x')

plt.show()
\end{lstlisting}

\section*{Conclusiones}

Para el presente caítulo, se puede concluir el aprendizaje obtenido a través del análisis de funciones en Numpy, una útil librería de \lstinline|Python| que permite realizar diferentes acciones matématicas, principalmente a través del uso de \textit{Arrays} o Arreglos de números.

Gracias a esta librería, a su vez, se pueden destacar una variedad de funciones, como se muestra para la primera sección de este capítulo, donde se abordaron funciones elementales como \lstinline|numpy.linspace| o \lstinline|numpy.random.normal|.

A su vez, se usaron las funciones analizadas en una serie de contextos a modo de problema para las posteriores secciones, donde se evaluaron respectivos usos, tales como la generación de números y aplicaciones en productos de diferente índole, y posibles limitaciones que presentan estos objetos a la hora de trabajar con arreglos de tipo numérico (para contextos de números reales o complejos), de esta forma, se pudo comprender de mejor manera los usos posibles y aplicaciones que tienen estas en el contexto de estudio para la asignatura o próximos proyectos.

Debido a esto, se espera que para próximos capítulos, se puedan utilizar las propiedad pertenecientes a las funciones analizadas con mayor profundidad en el ámbito del cálculo numérico.

\section*{Agradecimientos}
Para este capítulo, se requirió ayuda de múltiples fuentes de documentación, especialmente para la librería de \lstinline|Numpy|, por lo cual se agradece la propia documentación que está posee.
Notar la mayoritaria creación del capítulo por Ignacio Falcón, además, notar el trabajo de Alvaro Osses en consejos y correcciones para el código de los problemas y a Joaquín Parra a la hora de desarrollar la tabla inicial, especialmente debido a los múltiples errores que se obtenían al momento de proceder a compilar el documento en cuestión.

Finalmente, se agradece al Dr. Roberto Navarro por las correcciones hechas para este capítulo, las cuales se analizaron en búsqueda de la mejor solución posible.
\end{document}
