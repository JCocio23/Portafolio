\documentclass[../portafolio.tex]{subfiles}

\begin{document}

% NO EDITE ESTE ARCHIVO

\chapter*{Introducción}
\addcontentsline{toc}{chapter}{Introducción}
\markboth{Introducción}{Introducción}

Un portafolio es una herramienta que recopila de manera organizada
evidencias del trabajo y aprendizaje del/la estudiante. Su propósito
es mostrar el proceso de construcción del conocimiento del/la
estudiante, permitiendo valorar su progreso, reflexión crítica y
desarrollo de competencias. Este portafolio, en particular, recopila
evidencias de aprendizaje de la asignatura de \textbf{Física
  Computacional II} (510240), dictada en el segundo semestre de 2025
en el departamento de Física de la \textbf{Universidad de Concepción}.

\medskip

Esta asignatura se enfoca en la resolución de problemas en Física
usando métodos numéricos y el lenguaje de programación
\texttt{python}. Al finalizar este portafolio, se espera que los y las
estudiantes logren:
\begin{enumerate}
\item Aplicar herramientas computacionales en la resolución
  numérica de problemas en Física.
\item Generar programas computacionales basados en algoritmos y
  conceptos de la física matemática y estadística.
\item Diferenciar, integrar y resolver ecuaciones diferenciales
  ordinarias, en forma numérica.
\end{enumerate}

\medskip

La evaluación consiste en este portafolio, \textbf{alojado en un repositorio} de
\href{https://github.com}{GitHub} en la organización
\href{https://github.com/fiscomp2-UdeC2025}{fiscomp2-UdeC2025}
especialmente habilitada para esta asignatura.

\begin{enumerate}
\item El documento debe ser escrito completamente en \LaTeX, y debe
  contener: (a) La identificación del/la estudiante en la portada y
  sección de información personal; (b) las evidencias de aprendizaje,
  que corresponden a la resolución de problemas asignados en las
  tareas del curso, y; (c) una sección de conclusiones final donde
  el/la estudiante reflexione sobre su proceso de aprendizaje.

\item Cada tarea, compuesta por 1 a 4 problemas, será publicada con
  plazo aproximado de dos semanas para su entrega en
  GitHub. Estas tareas deben ser resueltas en grupos de 3 estudiantes
  (2 en casos debidamente justificados). Los grupos no son fijos y podrán
  reorganizarse en cada nueva tarea.

\item La solución de cada problema debe ser presentada en un capítulo
  independiente del portafolio (un problema = un capítulo). Cada
  capítulo debe contener: (a) los \textbf{autores} de la tarea; (b) un
  \textbf{resumen} breve de los objetivos; (c) un \textbf{desarrollo}
  de la solución explicado de forma clara, completa y concisa; y (d)
  una sección de \textbf{agradecimientos} donde se indique cuál fue el
  \textbf{aporte de cada integrante del grupo} y, si corresponde, una
  declaración sobre el uso de herramientas de IA o de ayuda externa al
  grupo.

\item Aunque las tareas se resuelvan en grupos, \textbf{cada
    estudiante es responsable de mantener su propio portafolio},
  asegurándose de la integridad del mismo (por ejemplo: entregar las
  soluciones antes del plazo límite, comprobar que el documento compila
  correctamente y sin errores, y redactar mensajes de
  \textit{commits} de forma  clara, descriptiva y adecuada).

\item La evaluación se realizará sobre la versión de uno de los
  integrantes del grupo, elegida aleatoriamente tras el cierre del
  plazo de cada tarea. La calificación y retroalimentación obtenidas
  serán válidas para todo el grupo.  Si un integrante del grupo no ha
  subido sus respuestas al momento del cierre, su calificación
  individual será la mínima, sin afectar la del resto de su
  grupo. Luego, se procederá a evaluar el portafolio de otro
  integrante para determinar la calificación del grupo.

\item Cada estudiante podrá editar su respuesta original para incluir
  esta retroalimentación en su propio portafolio, de manera
  independiente del resto del grupo.  Esto generará una nueva
  evidencia, cuyo progreso puede ser seguido con el historial de
  git/github. \textbf{No está permitido eliminar ni alterar este
    historial}, ya que es la única forma de evidenciar el proceso de
  aprendizaje.

\item El portafolio completo será evaluado al final del semestre, de
  forma independiente de las tareas, lo que constituirá la nota de
  portafolio. Cada estudiante tendrá plazo hasta el \textbf{miércoles
    10 de diciembre de 2025} para incorporar en el documento final las
  mejoras sugeridas en la retroalimentación de las tareas
  parciales. En esta revisión se evaluará: (a) que cada problema haya
  sido resuelto de forma completa y correcta; (b) que el/la estudiante
  haya incorporado las correcciones sugeridas en la retroalimentación; y (c) la
  coherencia y calidad del capítulo de conclusiones.

\item Se considerará causal de reprobación inmediata de la asignatura,
  con sanciones académicas correspondientes, cualquiera de las
  siguientes conductas:
  \begin{enumerate}
  \item Plagio total o parcial del portafolio, ya sea entre
    estudiantes del mismo curso o a partir de fuentes externas. En
    caso de detectarse copia sustancial entre portafolios, ambos
    estudiantes involucrados serán sancionados, salvo que existan
    evidencias claras que identifiquen al autor original.

  \item Presentar un portafolio no auténtico, elaborado por otra
    persona o generado de manera automática mediante inteligencia
    artificial u otros medios sin participación real o sin declaración
    explícita del estudiante.

  \item Entrega del portafolio por medios distintos a la plataforma
    oficial definida para el curso.

  \item Alteración o falsificación de evidencias, incluyendo
    manipulación de resultados, datos o retroalimentación docente, con
    el fin de obtener ventaja académica indebida.
  \end{enumerate}
\end{enumerate}

\subsection*{Criterios de evaluación}
Como criterios de evaluación general, se considerará:
\begin{enumerate}
\item Coherencia de las evidencias con los resultados de aprendizaje del curso y la autoevaluación al final del documento.
\item Redacción y competencias comunicativas, incluyendo ortografía,
  gramática, sintaxis, presentación de figuras, uso de
  \LaTeX\ y explicación clara de la parte relevante de scripts de
  \texttt{python}.
\item Presentación: Claridad, limpieza y orden del documento.
\item Incorporación de retroalimentación recibida en entregas
  parciales y evidencia de progreso a lo largo del curso.
\end{enumerate}

\medskip

Este portafolio, además de ser una herramienta de evaluación,
es una oportunidad para que el/la estudiante reflexione
sobre su proceso de aprendizaje, consolide sus conocimientos y
desarrolle habilidades clave en el ámbito de la computación aplicada a
la física. Se espera que este documento sirva como un registro
tangible de su progreso y de las competencias
adquiridas en la asignatura, las cuales podrán ser de gran
utilidad en futuros desafíos académicos y profesionales.

\end{document}
