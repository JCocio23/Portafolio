\documentclass[../portafolio.tex]{subfiles}

\begin{document}

\chapter*{Conclusiones}
\addcontentsline{toc}{chapter}{Conclusiones}
\markboth{Conclusiones}{Conclusiones}

\hfill \textbf{Fecha de presentación:} Jueves 12 de diciembre de 2025

\medskip

%--------------------------------------------------------------------------------
% Inicie con un resumen breve de cuáles eran los objetivos del portafolio;

%--------------------------------------------------------------------------------
% [Resumen de los contenidos]
% - Un resumen MUY breve de cuáles son las evidencias de aprendizaje que incluyó en este portafolio. Algo como "En el capítulo 1, se derivó numéricamente la función coseno, usando un esquema de derivadas centradas, para estudiar el error absoluto con respecto a la derivada analítica de la misma función."
% - Incluya una breve reflexión de lo que aprendió en cada actividad, lo que faltó aprender, lo que no se entendió y lo que sí se entendió bien.
% - Haga lo anterior por cada evidencia de aprendizaje.


%--------------------------------------------------------------------------------
% [Autoevaluación del alumno/a]
% Realice una reflexión de cómo trabajó usted, qué cree haber hecho bien y mal en el curso, qué le gustaría  hacer a futuro (en la forma de estudiar y en cómo cree que aplicará los contenidos de este portafolio en el futuro), cómo han distribuido su trabajo a lo largo del trabajo en este portafolio.

% --------------------------------------------------------------------------------
% [Evaluación del curso]
% - Realice una comparación entre sus expectativas iniciales, tal como las describió en la sección de presentación, y lo que realmente aprendió y experimentó a lo largo del curso. ¿Se cumplieron sus expectativas sobre la asignatura? ¿En qué medida?
% - ¿Qué aspectos del curso o del portafolio superaron, igualaron o no alcanzaron sus expectativas iniciales?
% - Agregue sugerencias para futuras versiones del curso, para que estudiantes de generaciones venideras se beneficien de una aplicación mejorada de este instrumento de evaluación.
% - ¿Cuál es la evidencia de este portafolio que usted cree es mejor/más relevante/en la que aprendió mejor? ¿Qué diferencia a esa evidencia del resto incluido en este portafolio?
% - ¿Puede evaluar la utilidad de este portafolio?

El objetivo de este portafolio siempre fue el de llevar registro de
los aprendizajes adquiridos durante este semestre y ser una forma
autónoma de poder mirar hacia atrás en el tiempo y reflexionar sobre los 
contenidos de esta asignatura.
Durante los once capítulos de este portafolio, quedan expresados
(en su mayoría) los contenidos del curso impartidos durante el 
segundo semestre de 2025, desarrollados por el autor y sus 
compañeros de trabajo. En el primer capítulo (\ref{ch:zen}), nos familiarizamos con 
\lstinline!python! descifrando un mensaje encriptado del mismo, haciendo 
uso de herramientas nativas. Luego, en el segundo capítulo (\ref{ch:analisis_funciones_numpy}), en la misma linea
nos familiarizamos con \lstinline!numpy!, una biblioteca fundamental para la 
computación científica de la cual utilizamos sus funciones más conocidas.
Cerrando la linea de conocer \lstinline!python!, en el tercer capítulo (\ref{ch:catalán}) aplicamos
lo aprendido en los capítulos anteriores para el estudio de los números de Catalán,
realizando cómputos y estimaciones de estos. \\
Comenzando otra línea temática, en el cuarto capítulo (\ref{ch:derivada_tres_puntos}) estudiamos una derivada de tres puntos no equidistantes, demostrando el error asociado a su aproximación y el orden del mismo, como también
su aplicación con nodos de Gauss-Chebyshev. Luego, en el capítulo cinco (\ref{ch:lotka}), estudiamos
la dinámica de poblaciones mediante el modelo de Lotka-Volterra, haciendo uso de métodos numéricos para ecuaciones diferenciales (en particular, leap-frog) para 
estudiar el comportamiento de cazadores y presas a la vez que aprendíamos como normalizar ecuaciones mediante el Teorema Pi de Buckingham. Finalmente en el capítulo seis (\ref{ch:shu}), 
implementamos otros método numérico para ecuaciones diferenciales, el famoso método Runge-Kutta para solucionar un problema de cinemática donde también se aplicó el Teorema Pi.\\
Abriendo otra linea temática, en el capítulo siete (\ref{cap:metodo_secante}) aprendimos como implementar en \lstinline!python! 
el método de la secante para hallar ceros, como también analizamos el error asociado a este método. Continuando, en el capítulo ocho (\ref{ch:euimp}) estudiamos el método implícito de Euler junto al método de Newton-Raphson para hallar soluciones numéricas a cierto
tipo de ecuaciones diferenciales. Finalmente, en el capítulo nueve (\ref{ch:spring}), implementamos el método de la bisección para estudiar la estabilidad de un péndulo invertido conectado a un resorte.\\
Cerrando con los contenidos del curso, en el capítulo diez (\ref{ch:polinomios_legendre}) se implementaron métodos de integración, como la regla de Simpson 1/3 compuesta, junto al método de interpolación de Lagrange para realizar un análisis y aproximación de los conocidos polinomios de Legrende. Finalmente, en el capítulo once (\ref{ch:co2}) implementamos un método de mínimos cuadrados para ajustar un modelo que busca modelar datos reales de concentración promedio de dióxido de carbono en la atmósfera.\\
En relación al trabajo personal, si bien se cometieron errores señalados por el
docente en las retroalimentaciones, considero que fui constante durante la 
creación de los capítulos y gracias a esas retroalimentaciones pude adaptarme
a una forma más profesional y estandarizada de redactar en ciencias.
Reconozco que en algunas ocasiones (a excepción de una que se comenta más abajo),
se dejó la elaboración de los capítulos para un par de día antes de la fecha de 
entrega de las tareas dado que en la mayoría de ocasiones existían evaluaciones cercanas en otras asignaturas
y se decidió aplazar el portafolio al ser más práctico, lo cual pudo haber afectado a la redacción y profundidad de
las respuestas entregadas. En un futuro, me gustaría poder organizar mejor mis tiempos y corroborar mis resultados. \\

Mis expectativas sobre la asignatura fueron sobrepasadas, pues si bien sabía que por fin iba a introducirme al mundo de
los métodos numéricos, pensé que el enfoque iba a ser mucho más práctico, quizás algo más cercano a la asignatura de 
Cálculo Numérico que toman en ingeniería. Para mi sorpresa, esta asignatura tuvo un fuerte componente teórico, que si bien
a ratos fue difícil, hizo la asignatura mucho más agradable para mi. Percibo que cierro esta asignatura con una amplio espectro
de métodos para afrontar los distintos problemas que vengan durante la carrera, al menos hasta que llegue Física Computacional 3.
Lo que más destaco es el capítulo acerca de la dinámica de poblaciones, el cual fue escrito mayoritariamente por mi. Ese capítulo
marcó un antes y un después para mi en esta asignatura, pues no solamente disfrute estudiando el método de salto de rana, si no que
una cadena involuntaria de clicks durante la investigación para realizar el capítulo me llevó a realizar el desarrollo extra que posee
ese capítulo. Recuerdo que cuando encontré la función $H(p,d)$ y noté que era igual a la función propuesta $C(p,d)$ de la tarea,
fue el momento más eufórico del semestre. Fue un capítulo que de verdad disfruté mucho hacer (también se dio que no habían otras
evaluaciones por aquellas fechas) y me motivó a estudiar y darle más tiempo a esta asignatura, independiente de como retornasen las 
futuras notas.\\

Finalmente, agradecer al docente del curso, Dr. Roberto Navarro y los ayudantes del curso por llevar la asignatura de una forma que personalmente se me hizo entretenida y pedagógica. 
Si bien a ratos el curso se hizo denso, las notas no fueron las mejores y me costó muchas veces aprender a implementar los contenidos en \lstinline!python!,
salgo de este curso con un buen sabor de boca y seguro que lo que aprendí será fundamental para el resto de mi carrera como científico. 
\end{document}
