\documentclass[../portafolio.tex]{subfiles}

% Solo agregue paquetes en el preámbulo de ../portafolio.tex

\begin{document}

\chapter*{Información personal y académica}
\addcontentsline{toc}{chapter}{Información personal y académica}
\markboth{Información personal y académica}{Información personal y académica}


%%%%%%%%%%%%%%%%%%%%%%%%%%%%%%%%%%%%%%%%%%%%%%%%%%%%%%%%%%%%%%%%%%%%%%
% Llene todos los campos, respetando tildes, mayúsculas y minúsculas.
\section*{Datos personales}

\begin{description}
\item[{Nombre completo}] Joaquín Ignacio Parra Sánchez % nombres y apellidos completos.
\item[{Matrícula}] 2024429400               % matrícula udec
  \item[{Fecha de Nacimiento}] 30 de Junio del 2005 % día de mes de año
\item[{Nacionalidad}] Chileno
  \item[{E-Mail institucional}] \href{mailto: jparra2024@udec.cl}{jparra2024@udec.cl}
\end{description}


%%%%%%%%%%%%%%%%%%%%%%%%%%%%%%%%%%%%%%%%%%%%%%%%%%%%%%%%%%%%%%%%%%%%%%
\section*{Breve biografía académica}
% Redacte una breve biografía (5 a 7 líneas) que incluya los
% siguientes aspectos:
% - Su nombre completo y el año en el que ingresó a la Universidad de
% Concepción.
% - Mencione su carrera actual y en qué año académico se encuentra.
% - Describa brevemente su trayectoria educativa previa a la universidad
% (por ejemplo, dónde cursó la educación media y cualquier logro académico
% relevante).
% - Mencione sus metas académicas y profesionales al finalizar el
% pregrado. ¿Qué le gustaría lograr al terminar la carrera? ¿En qué
% áreas le gustaría especializarse o trabajar?
% - Si lo considera pertinente, puede mencionar cualquier actividad
% extracurricular que haya contribuido a su formación (cursos,
% proyectos, trabajos, etc.).
Mi nombre es Joaquín Ignacio Parra Sánchez, estudiante de la carrera de Ciencias Físicas en la Universidad de Concepción, Chile. Ingresé a la carrera el año 2024 y me encuentro cursando el cuarto semestre. Previo a mi ingreso a la UdeC, estudié en el Colegio Seminario Padre Albero Hurtado de Chillán. Durante la enseñanza media participé activamente en clubes de debate de mi institución, y en olimpiadas de física y matemática, destacando haber obtenido primer lugar en las olimpiadas de física UdeC, medalla de oro y plata en las olimpiadas de matemática de la UBB y UFRO, respectivamente en 2023. Espero especializarme en algún área dentro del margen de gravitación y altas energías (aunque ya esté trillado).


%%%%%%%%%%%%%%%%%%%%%%%%%%%%%%%%%%%%%%%%%%%%%%%%%%%%%%%%%%%%%%%%%%%%%%
\section*{Visión general e interés sobre la asignatura}
% En esta sección, reflexione y describa:
% - ¿Cuál es su percepción inicial sobre la asignatura de Física
% Computacional II? ¿Cómo se relaciona con su formación académica y sus
% intereses?
% - ¿Qué habilidades o conocimientos espera desarrollar en esta
% asignatura, específicamente en el uso de herramientas computacionales
% aplicadas a la física?
% - ¿De qué manera cree que lo aprendido en esta asignatura contribuirá a
% su desempeño en otros cursos o en su carrera profesional a futuro?
% - Si tiene alguna expectativa específica o tema de interés particular
% dentro de la asignatura, menciónelo aquí.
Comencé mirando a Física Computacional II como un ramo ajeno pero intrigante, 
pues al igual que la mayoría de personas, no tuve un acercamiento previo a la programación durante mi adolescencia. En el futuro inmediato, espero que este curso
me entregue las herramientas básicas para abordar problemas que no posean soluciones analíticas, o que sean muy complicadas (como el péndulo doble). \\
Espero desarrollar competencias sólidas en programación y sobre todo, tener una base
lo suficientemente sólida para poder abordar otros problemas computacionales (por ejemplo, alguna EDP). Considero que este curso me entregará herramientas fundamentales para abordar problemas en los cursos de Mecánica Clásica I y Electrodinámica I, donde muchas veces los cálculos se vuelven demasiado complejos como para intentar conseguir la solución analítica.


%%%%%%%%%%%%%%%%%%%%%%%%%%%%%%%%%%%%%%%%%%%%%%%%%%%%%%%%%%%%%%%%%%%%%%
\section*{Resultados esperados de este portafolio}
% En esta sección, reflexione sobre los resultados que espera obtener al
% realizar este portafolio. Puede incluir lo siguiente:
% - ¿Qué habilidades y conocimientos espera haber consolidado al completar
% este portafolio?
% - ¿Cómo cree que el portafolio le ayudará a organizar, analizar y
% aplicar los conceptos aprendidos durante la asignatura?
% - ¿De qué manera considera que este portafolio puede servirle como
% referencia o herramienta para su futura formación académica o
% profesional?
% - Reflexione sobre cómo el proceso de autoevaluación y la inclusión de
% evidencias le permitirá comprender mejor su propio progreso.
Al finalizar este portafolio, espero haber desarrollado un manejo adecuado de herramientas matemáticas de nivel intermedio para aplicarlas al modelado y resolución de problemas de física donde no es posible obtener soluciones analíticas simples. La organización del portafolio me permitirá presentar de manera clara y ordenada la evidencia de mis aprendizajes, así como llevar un seguimiento de mi progreso a lo largo del semestre.


\end{document}
